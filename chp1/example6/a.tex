\documentclass{article}

\usepackage{blindtext}
\usepackage{xspace}

\newcommand{\Exnewcommand}{\textbackslash{}newcommand\{\}\xspace}
\newcommand{\commandEx}{\textbackslash{}command}
\newcommand{\VT}{Vermont}

\newcommand{\TUG}{\textsc{TeX Users Group}}


\begin{document}

\title{Example 6: White-space characters}
\date{\today}
\author{Jave}
\maketitle

% \section{What is xspace?}
% \noindent xspace allows commands to intuitively know if there needs a space at the end 
% of \Exnewcommand{}

% \noindent xspace allows spacing if there are alphabetical letters that come after a 
% command. If not, there won't be a space character.

% \vspace{2.0mm}

% This is without xspace: \commandEx XSPACE 

% THis is with xspace: \commandEx XSPACE 

Our college is in \VT.
The \VT summers are nice

\end{document}

% \usepackage{blindtext}
% \usepackage{xspace}

% \newcommand{\newcommandEx}{\textbackslash{}newcommand\{\}\xspace}
% \newcommand{\TUG}{\textsc{TeX Users Group}}

% \begin{document}

% \section{What is xspace?}

% xspace allows commands to intuitively know if there needs a space at the end 
% of \newcommandEx

% % \section{The \TUG}
% % The \TUG is an organization for people who are interested in \TeX or \LaTeX.



 
