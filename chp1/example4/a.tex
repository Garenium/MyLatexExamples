\documentclass[letterpaper, 12pt]{article}

\begin{document}

\newcommand{\textttEx}{\textbackslash{}texttt\{\}}
\newcommand{\textsfEx}{\textbackslash{}textsf\{\}}
\newcommand{\textrmEx}{\textbackslash{}textrm\{\}}

\newcommand{\rmfamilyEx}{\textbackslash{}rmfamily}
\newcommand{\sffamilyEx}{\textbackslash{}sffamily}
\newcommand{\ttfamilyEx}{\textbackslash{}ttfamily}

%These commands allow changing the default font throughout the entire page
%the moment they are called
%\sffamily
%\rmfamily

\title{Example 4: Fonts}
\date{\today}
\author{Jave}
\maketitle

\section{\textsf{\LaTeX\ resources (\textsfEx) \textrm{on the internet (\textrmEx)} } }

The best place for downloaded LaTeX related software is CTAN\@.
Its address is \texttt{http://www.ctan.org} (\textttEx)\\

If you don't want to specify where in the text you wish to change the font, these commands are needed
\begin{itemize}
  \item \rmfamily Roman font (\rmfamilyEx)
  \item \sffamily Sans-serif font (\sffamilyEx)
  \item \ttfamily Typewriter font (\ttfamilyEx)
\end{itemize}

If you do want to specify where in the text, use these. 
\begin{itemize}
  \item \textrm{Roman font} (\textrmEx)
  \item \textsf{Sans-serif font} (\textsfEx)
  \item \texttt{Typewriter font} (\textttEx)
\end{itemize}

\noindent To experiment the three pre-defined family fonts (the first three):

  \subsection{The Roman font (default font with \rmfamilyEx)}
  \rmfamily
Lorem ipsum dolor sit amet, consectetur adipiscing elit. Duis non mi id justo iaculis lobortis vitae sed ligula. Nullam a ipsum aliquet, malesuada mi sit amet, mollis massa.
Nulla ultrices libero vel tortor pulvinar, in viverra velit sagittis. Praesent vestibulum efficitur dignissim. Sed nec urna accumsan metus venenatis fermentum. 
Donec semper consequat nibh. 

\ttfamily
  \subsection{\ttfamily The typewriter font (\ttfamilyEx)} 
Lorem ipsum dolor sit amet, consectetur adipiscing elit. Duis non mi id justo iaculis lobortis vitae sed ligula. Nullam a ipsum aliquet, malesuada mi sit amet, mollis massa.
Nulla ultrices libero vel tortor pulvinar, in viverra velit sagittis. Praesent vestibulum efficitur dignissim. Sed nec urna accumsan metus venenatis fermentum. 
Donec semper consequat nibh. 

\sffamily 
  \subsection{The Sans-serif font (\sffamilyEx)} 
  \textbf{Notice that the subsection is not in sans-serif because (\sffamilyEx) is out of scope}
Lorem ipsum dolor sit amet, consectetur adipiscing elit. Duis non mi id justo iaculis lobortis vitae sed ligula. 
Nullam a ipsum aliquet, malesuada mi sit amet, mollis massa. Nulla ultrices libero vel tortor pulvinar, in viverra velit sagittis. Praesent vestibulum efficitur dignissim. 
Sed nec urna accumsan metus venenatis fermentum. Donec semper consequat nibh.\




\end{document}


