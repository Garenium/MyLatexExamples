\documentclass[letterpaper, 12pt]{article}

\begin{document}
\title{Example 3}
\author{jave}
\date{\today}
\maketitle

%notice how every text-related commands that change the "looks" start with text***?
%\emph uses italic by default
%\emph{\emph} goes back to normal (it's just made that way I guess)
%\textsc means slanted text
%This is the old one, it looks ugly
% Text can be \emph{emphasized} (\textbackslash{}emph\{\}).
% Besides being \textit{italic} (\textbackslash{}textit\{\}) words could be \textbf{bold}, (\textbackslash{}textbf\{\})
% \textsl{slanted} (\textbackslash{}textsl\{\}) or typeset in \textsc{Small Caps} (\textbackslash{}textsc\{\}).
% Such commands can be \textit{\textbf{nested}} (\textbackslash{}texit\{\textbackslash{}textbf\{\}\}).
% \emph{See how \emph{emphasizing} looks when nested.} (\textbackslash{}emph\{\textbackslash{}emp\{\}\})

%we can instead make our own commands (shorthand-commands) with the command /newcommand
%the ending _ex stands for "example"
\newcommand{\emphEx}{\textbackslash{}emph\{\})}
\newcommand{\textitEx}{\textbackslash{}textit\{\}}
\newcommand{\textbfEx}{\textbackslash{}textbf\{\}}
\newcommand{\textslEx}{\textbackslash{}textsl\{\}}
\newcommand{\textscEx}{\textbackslash{}textsc\{\}}

Text can be \emph{emphasized} (\emphEx).
Besides being \textit{italic} (\textitEx) words could be \textbf{bold}, (\textbfEx)
\textsl{slanted} (\textslEx) or typeset in \textsc{Small Caps} (\textscEx).
Such commands can be \textit{\textbf{nested}} (\textbackslash{}texit\{\textbackslash{}textbf\{\}\}).
\emph{See how \emph{emphasizing} looks when nested.} (\textbackslash{}emph\{\textbackslash{}emp\{\}\})

\end{document}
\end{document}
